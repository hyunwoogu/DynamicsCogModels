
\documentclass[12pt]{article} 

\usepackage{geometry}
\geometry{a4paper} 

\usepackage{graphicx} 

\usepackage{float} 
\usepackage{wrapfig} 

\usepackage{amsmath}
\usepackage{amsfonts}
\usepackage{amssymb}
\usepackage{dsfont}

\linespread{1.2} 

\setlength\parindent{0pt} % Uncomment to remove all indentation from paragraphs


\begin{document}

\title{\textbf{Derivation of Simple Harmonic Motion Equation}}
\author{Hyunwoo Gu}
\date{}

\maketitle

%----------------------------------------------------------------------------------------
%	Prerequisites
%----------------------------------------------------------------------------------------
\section*{Prerequisites}

\subsection{Characteristic equation}

\textbf{Characteristic equation} is an algebraic equation of degree $n$ upon which depends the solution of a given $n$th-order differential equation. 

For example, for the following DE problem,

$$
y'' - 4y' + 3y = 0
$$

by change of function (which is valid due to \textbf{Picard–Lindelöf theorem of initial value problems}),

$$
y(t) := c \dot e^{\theta t}
$$

we can obtain

$$
\theta^2 - 4\theta + 3 = 0
$$

By solving for $\theta$, we get the solution of DE. 

\subsection{Euler's formula}

$$
e^{ix} = \mathrm{cos}(x) + i \mathrm{sin}(x)
$$

\textbf{Proof}

By Maclaurin series expansion, 

$$
\begin{aligned}
e^{ix} &= 1 + ix + \frac{(ix)^2}{2!} + \frac{(ix)^3}{3!} + \frac{(ix)^4}{4!} + \frac{(ix)^5}{5!} \cdots \\[10pt]
&= \left( 1 - \frac{x^2}{2!} + \frac{x^4}{4!} + \cdots \right) + i\left(  x - \frac{x^3}{3!} + \frac{x^5}{5!}  + \cdots \right) \\[10pt]
&= \mathrm{cos}(x) + i \mathrm{sin}(x)
\end{aligned}
$$


\subsection{Product rule}

$$
\frac{d}{dx} \left( u \dot v \right) = \left( \frac{du}{dx} \right) v + u  \left( \frac{dv}{dx} \right)
$$

\scripsize
(Reference: https://en.wikipedia.org/wiki/Product_rule)
\normalsize

\subsection{Chain rule}

$$
\frac{dz}{dx} = \frac{dz}{dy} \dot \frac{dy}{dx}
$$

\scripsize
(Reference: https://en.wikipedia.org/wiki/Chain_rule)
\normalsize


%----------------------------------------------------------------------------------------
%	Simple Solution
%----------------------------------------------------------------------------------------
\pagebreak
\section*{Basic approach}

$$
m \ddot{x} + k x = 0
$$

The \textbf{characteristic equation} for the above ODE is 

$$
m \theta^2 + k = 0
$$ 

which has complex roots

$$
\theta = \pm i \sqrt{k/m}
$$

Thus, the solution is given by

$$
\begin{aligned}
x(t) &= c_1 e^{i \omega t} + c_2 e^{-i \omega t} \\[8pt]
&= c_1 (\mathrm{cos} (\omega t) + i \mathrm{sin} (\omega t)) + c_2 (\mathrm{cos} (\omega t) - i \mathrm{sin} (\omega t)) \\[8pt]
&= d_1 \mathrm{cos} (\omega t) + d_2 \mathrm{sin} (\omega t) \\[8pt]
&= A \mathrm{cos} (\omega t + \delta)
\end{aligned}
$$

where $\omega := \sqrt{k/m}$. Note that $d_1, d_2 \in \mathbb{R}$. The last line is due to the fact that the \textbf{linear combination of sinusoids are again sinusoids}, but with a new amplitude and phase shift.

Interpretation : $\omega = \sqrt{k/m}$ as \textbf{frequency}, $A = \sqrt{d_1^2 + d_2^2} $ as \textbf{amplitude}, and $\delta$ as \textbf{phase angle}.  



\pagebreak
%----------------------------------------------------------------------------------------
%	Advanced approach
%----------------------------------------------------------------------------------------

\section*{Advanced approach}

$$
m \ddot{x} + k x = 0
$$

By multiplying $\dot{x}$ to the above equation, we can see

$$
\begin{aligned}
m \ddot{x} \dot{x} + k x \dot{x} &= 0 \\[10pt]
m \frac{d}{dt} \left( \frac{dx}{dt} \right) \dot \frac{dx}{dt} + k x \frac{dx}{dt} = 0
\end{aligned}
$$

Since 

\begin{itemize}
	\item $\frac{d}{dt} \left( \frac{dx}{dt} \right) \dot \frac{dx}{dt} = \frac{1}{2} \left[ \frac{d}{dt} \left( \frac{dx}{dt} \right) \dot \frac{dx}{dt} + \frac{dx}{dt} \dot  \frac{d}{dt} \left( \frac{dx}{dt} \right) \right] = \frac{d}{dt} \left( \left( \frac{dx}{dt} \right)^2 \right) $
	\item $x \frac{dx}{dt} = \frac{d (x^2/2) }{dx} \dot \frac{dx}{dt} = \frac{d (x^2/2)}{dt}$
\end{itemize}

we get the following \textbf{energy conservation law} formula, with the first term being the kinetic energy, and the second term being the potential energy of the spring.

$$
\begin{aligned}
\int \frac{m}{2} \frac{d}{dt} \left( \left( \frac{dx}{dt} \right)^2 \right) dt + \int k \frac{d (x^2/2)}{dt} dt &= E\\[10pt]
m \frac{(\dot{x})^2}{2} + k\frac{x^2}{2} &= E \\[12pt]
\end{aligned}
$$

If we proceed, 

$$
\begin{aligned}
m \frac{(\dot{x})^2}{2} + k\frac{x^2}{2} &= E \\[12pt]
\dot{x} &= \pm \sqrt{\frac{2E}{m} - \frac{k}{m} x^2 } \\[10pt]
\frac{dx}{\sqrt{\frac{2E}{m} - \frac{k}{m} x^2 }} &= \pm dt \\[10pt]
\mathrm{arccos} x \sqrt{k}{2E} &= \sqrt{\frac{k}{m}}t + \phi \\[10pt]
x &= \sqrt{\frac{2E}{k}} \mathrm{cos}(\omega t + \phi)
\end{aligned}
$$

since $\frac{d}{dx} \mathrm{arccos}(x) = \frac{1}{ \sqrt{1 - x^2}}$


(Reference : https://en.wikibooks.org/wiki/Ordinary_Differential_Equations/Simple_Harmonic_Motion)


\end{document}